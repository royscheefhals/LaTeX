\begin{document}
Aangezien de opdracht zelf bedacht is en meer een richting specificeert, moeten de harde requirements nog wel beschreven worden. Tijdens het maken van het plan van aanpak is dan ook vooronderzoek gedaan naar het onderwerp. In dit hoofdstuk wordt het vooronderzoek dat is uitgevoerd beschreven.

\section{Bestaande systemen}
Ander vooronderzoek dat is gedaan is het onderzoek naar bestaande systemen gerelateerd aan de opdracht. 

Een ander branddetectiesysteem gebruik makend van WSN, is FireGrid\cite{FireGrid}. FireGrid gebruikt WSN in gebouwen in combinatie met Grid Computing om gebouwen of omgevingen te monitoren. Dit doen zij alleen wel via een gecentraliseerde oplossing. In die opstelling zal er dan een high performance computer zijn die alle sensor data verzamelt en de benodigde simulaties uitvoert. In een omgeving die robuust moet zijn in gevaarlijke omgevingen is dat een probleem. Als er namelijk netwerk problemen zijn op die locatie, of de brand bevind zich op die locatie, kan het gehele systeem uitvallen. 

%NEMBES
%FINDER
%FireNet?
%die ene powerpoint

\section{Haalbaarheid Uitgangspunt}
Het idee om een gedistribueerd WSN te gebruiken die helpt bij de navigatie van personen bij een noodsituatie is een zelf bedacht idee. Om te controleren of dit een waardig idee is, is er onderzoek gedaan naar eerder uitgevoerde onderzoeken naar dit idee.

Het artikel 'Emergency Evacuation using Wireless Sensor Networks'\cite{BarnesEmEv} beschrijft het idee om een gedistribueerd WSN te gebruiken bij evacuatie. Het artikel beschrijft een algoritme die een route berekend naar bepaalde uitgangspunten binnen een netwerk, rekening houdend met het vermijden van de positie van de positie van het gevaar.

Dit artikel beschrijft grotendeels de set aan requirements die gedefinieerd zijn. Het beschrijft de benodigdheid om een decentraal WSN te hebben

\end{document}