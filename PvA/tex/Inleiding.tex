\documentclass{../local}
\begin{document}

In dit document wordt het plan van aanpak beschreven voor de opdracht Gedistribueerde Hardware, wat initieel staat beschreven in het afstudeervoorstel. Het plan van aanpak moet een gedetailleerde specificatie zijn van wat er inhoudelijk moet worden gedaan en opgeleverd worden.

Allereerst wordt er beschreven wat de aard is van de opdracht. Zoals in het afstudeervoorstel is aangegeven, is de opdracht niet direct een oplossing voor een problem, maar een te benutten kans. Het onderwerp is wel zo gekozen dat deze in de interesse valt bij de afstudeerder en dat Alten PTS er een demonstratie product aan over houdt. 

Doordat de opdracht op een 'creative wijze' is gekozen, wordt er wel eerst onderzocht of de gekozen richting wel haalbaar is. Dat wordt beschreven in hoofdstuk Vooronderzoek.

Met het resultaat uit dit kleine onderzoek, worden de requirements met daarbij horende onderzoeksvragen opgesteld. Deze kunnen op dat moment met meer zekerheid worden opgesteld. Als laatste wordt de werkwijze beschreven met planning en risicoanalyse.

Dit document wordt ondertekend door middel van een afstudeercontract. Bij ondertekening wordt verklaard dat de docentbegeleider en bedrijfsbegeleider akkoord gaan met de inhoud van dit plan van aanpak.

\end{document}