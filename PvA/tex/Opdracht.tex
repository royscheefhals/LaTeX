

\documentclass{../local}

\begin{document}
In voorgaande hoofdstuk is de aard van de opdracht besproken. In dit hoofdstuk zal de opdracht zijn doelstellingen en requirements worden gedefinieerd. Hieraan worden ook nog onderzoeksvragen opgesteld met bijbehorende deelvragen.

\section{Doelstellingen}
Het doel van dit project is om een demonstrator te hebben voor Alten PTS die getoond kan worden aan potentiële klanten die geïnteresseerd zijn in oplossingen voor veiligheid op de werkvloer. Voor dit project geldt alleen een product wat binnen dit onderwerp gebruikt kan worden, namelijk het concept om een Wireless Sensor Netwerk (WSN) te gebruiken bij het evacueren van personen in een noodsituatie uitwerken.

Hoewel dit project een onderdeel is van een overkoepelend project met zijn eigen doelstelling voor Alten PTS, is het goed om een doelstelling te defineren voor het product zelf. De doelstelling van dit product is om bij een noodsituatie, waarbij een gebouw moet worden ontruimd, de personen binnen een gebouw op intuïtieve wijze een route te geven naar de dichtstbijzijnde nooduitgang. 

Als uitbreiding op deze doelstelling kan het WSN na de evacuatie de reddingswerkers dienen door actief sensor informatie door te sturen naar mogelijk mobiele ontvangers. Hierdoor is het mogelijk dat de reddingswerker een verwachting heeft over de komende situatie, of een ideale route geven naar een bepaalde locatie. Ook kan de reddingswerker tegelijkertijd gevolgd worden vanaf een basisstation \cite{ShaWSN}.

\section{Requirements}
Om de doelstelling te behalen zijn er requirements opgesteld. Deze zijn opgesteld volgens het MoSCoW methode. 
\\\\
\noindent\textbf{Must Have}

\begin{itemize}
\item Het hebben van een WSN. Dit netwerk moet dienen als een branddetectiesysteem.

\begin{itemize}
\item De individuele onderdelen binnen een WSN (nodes) hoeven niet een hoogwaardig branddetectiesysteem te hebben, voor deze demonstratie is een schakelaar of knop genoeg. 
\end{itemize}

\item Het WSN moet bij een brandmelding een route naar de uitgang tonen die de gevaarlijke zones vermijdt.

\begin{itemize}
\item Voor de nodes hoeft de aanduiding van de route niet meteen intuïtief te zijn. Aangeven door middel van LED's kan genoeg zijn.
\end{itemize}

\item Het WSN moet een robuust netwerk zijn. Dat wil zeggen dat wanneer er een node uitvalt binnen het WSN, het volledige systeem niet mag uitvallen.

\end{itemize}

\noindent\textbf{Should Have}
\begin{itemize}
\item Meerdere soorten gevaren detecteren dan alleen brandgevaar. Hiervoor kan een abstractielaag voor worden ontworpen.
\end{itemize}
\noindent\textbf{Could Have}
\begin{itemize}
\item De nodes van het WSN binnen het gebouw moet data overbrengen naar een speciale 'handheld' van de reddingswerkers als die in de buurt is.
\end{itemize}
\noindent\textbf{Won't Have Now}
\begin{itemize}
\item Voorspelling van de verspreiding van het gevaar (brandverspreiding, gasverplaatsing).
\item makkelijke integratie door middel van bestaande vluchtroutes.
\end{itemize}
\section{Onderzoeksvragen}

\section{Resultaten en Eindproducten}

\section{Aannames}
\end{document}