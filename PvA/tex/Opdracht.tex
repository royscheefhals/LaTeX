

\documentclass{../local}

\begin{document}
In voorgaande hoofdstuk is bewezen dat het onderwerp van deze opdracht haalbaar is. In dit hoofdstuk wordt met deze informatie de opdracht gespecificeerd door middel van doelstellingen, requirements, onderzoeksvragen, en verwachtte eindproducten op te stellen.

\section{Doelstellingen}
Het doel van dit project is om een demonstrator te hebben voor Alten PTS die getoond kan worden aan potentiële klanten die geïnteresseerd zijn in oplossingen voor veiligheid op de werkvloer. Voor dit project geldt alleen een product wat binnen dit onderwerp gebruikt kan worden, namelijk het concept om een Wireless Sensor Netwerk (WSN) te gebruiken bij het evacueren van personen in een noodsituatie uitwerken. De demonstrator moet wel werkend zijn op fysieke hardware.

Hoewel dit project een onderdeel is van een overkoepelend project met zijn eigen doelstelling voor Alten PTS, is het goed om een doelstelling te defineren voor het product zelf. De doelstelling van dit product is om bij een noodsituatie, waarbij een gebouw moet worden ontruimd, de personen binnen een gebouw op intuïtieve wijze een route te geven naar de dichtstbijzijnde nooduitgang. 

Als uitbreiding op deze doelstelling kan het WSN na een evacuatie aanwezige reddingswerkers te dienen door actief sensor informatie door te sturen naar mobiele ontvangers. Hierdoor is het mogelijk dat de reddingswerker een verwachting heeft over de komende situatie, of een ideale route kan geven naar een bepaalde locatie. Ook kan de reddingswerker tegelijkertijd gevolgd worden vanaf een basisstation \cite{ShaWSN}.

\section{Requirements}
Om de doelstelling te behalen zijn er requirements opgesteld. Deze zijn opgesteld volgens de MoSCoW methode. 
\\\\

\noindent\textbf{Must Have}
\begin{itemize}
\item Het WSN moet dienen als een brandmeldsysteem.
\begin{itemize}
\item De individuele onderdelen binnen een WSN (nodes) hoeven niet een hoogwaardig branddetectiesysteem te hebben, voor deze demonstratie is een schakelaar of knop genoeg.
\end{itemize}
\item Het WSN moet bij een brandmelding een route naar de uitgang tonen die de gevaarlijke zones vermijdt.
\begin{itemize}
\item Voor demonstratie doeleinden hoeft de aanduiding van de route d.m.v. de nodes niet direct intuïtief te zijn. Aangeven door middel van LED's kan genoeg zijn.
\end{itemize}
\item Het WSN moet een robuust netwerk zijn. Dat wil zeggen dat wanneer er een node uitvalt binnen het WSN, het volledige systeem niet mag uitvallen.
\end{itemize}

\noindent\textbf{Should Have}
\begin{itemize}
\item Meerdere soorten gevaren detecteren dan alleen brandgevaar. Hiervoor kan een abstractielaag voor worden ontworpen.
\item Een systeem die de status van de WSN nodes kan monitoren. Te monitoren waardes zullen zijn:
\begin{itemize}
\item Status van WSN node (operationeel, defect, etc.).
\item Aangegeven bepaalde navigatie wanneer de noodtoestand bezig is.
\end{itemize}
\end{itemize}

\noindent\textbf{Could Have}
\begin{itemize}
\item De nodes van het WSN binnen het gebouw moet data overbrengen naar een speciale 'handheld' van de reddingswerkers als die in de buurt is.
\item Gebruik van sensoren om gevaren te detecteren (temperatuursensoren, CO$_{2}$ sensoren). Dan ook het toevoegen van sensor waarden bij de monitor applicatie.
\end{itemize}

\noindent\textbf{Won't Have Now}
\begin{itemize}
\item Voorspelling van de verspreiding van het gevaar (brandverspreiding, gasverplaatsing).
\item makkelijke integratie door middel van bestaande vluchtroutes.
\item Het updaten van software is cruciaal voor bijna elk product. Voor WSN's kan updaten lastig zijn aangezien de nodes overal binnen een gebouw zich kunnen bevinden. Elke node met een draad aansluiten is inefficient en foutgevoelig. Een 'over the air update mechanisme' beschreven in \cite{StatRUP} is cruciaal. Voor demonstratie doeleinden is dit niet nodig, maar voor real-life implementaties is dit onmisbaar.
\end{itemize}
\section{Onderzoeksvragen}


\section{Resultaten en Eindproducten}


\section{Aannames}
Aangezien het product uiteindelijk een demonstratie model wordt, kunnen een aantal aannames gedaan worden om de ontwikkeling te versimpelen. Door de versimpeling kan er meer functionaliteit worden toegevoegd.

\begin{itemize}
\item Wireless sensor nodes zijn zo ontworpen dat deze op elke locatie kan worden geplaatst. Ook op locaties waar niet een vaste stroomvoorziening is. De Wireless sensor nodes moeten dan op batterij draaien. Bij het gebruik van wireless sensor nodes moet dan ook altijd het stroomverbruik in acht genomen worden. In dit project wordt dat niet gedaan. Er wordt aangenomen dat er altijd stroomvoorziening is.

\end{itemize}

\end{document}