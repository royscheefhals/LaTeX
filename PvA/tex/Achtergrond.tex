\documentclass{../local}
\begin{document}

\section{Context Afstudeeropdracht}
In dit hoofdstuk wordt de context van de opdracht beschreven. Hierbij komt het bedrijf aan de orde, waar de opdracht wordt uitgevoerd, wat de plaats van de student is binnen het bedrijf en of de opdracht raakvlakken heeft met andere projecten. 

\subsection{Het Afstudeerbedrijf: Alten PTS}
Alten is een wereldwijd bedrijf die bij elk project betrokken wil zijn waarbij technologie een hoofdzaak is en komt van origine uit Frankrijk. Alten PTS is de business unit binnen Alten Group die is gespecialiseerd in de technische automatisering. Alten PTS is gevestigd binnen Nederland.

Bij Alten PTS draait het om techniek. De opdrachtgevers van Alten zijn toonaangevende technisch georiënteerde bedrijven, onder andere in de industrie, telecom en verkeer en vervoersmarkt. Alten PTS heeft 200 medewerkers die gezamenlijk de volgende diensten bieden:

\begin{itemize}

\item Engineering \& Technology Consultancy: Het bieden van oplossingen voor problemen op basis van specifieke kennis en ervaring.
\item Eigen projecten en Research \& Development outsourcing: Het nemen van de verantwoordelijkheid over projecten en de verzorging ervan.
\item Training: Alten PTS is gespecialiseerd op het gebied van software ontwikkeling en geeft ook trainingen hierover.

\end{itemize}

Het bieden van de dienst Engineering \& Technology Consultancy is het grootste aandeel binnen Alten.\cite{Alten}

\subsection{Plaats student binnen Alten PTS}
De opdracht zal worden uitgevoerd op kantoor west van Alten PTS. Hier wordt een vaste werkplek aangeboden. Deze werkplek zal zich bevinden tussen andere werknemers van Alten PTS die op het moment geen opdracht hebben bij een extern bedrijf (Consultancy), of die op het moment zelf een project aan het uitvoeren zijn (bv. een Research en Development project).

Binnen Alten PTS zal de student als stagiair aan de slag gaan en heeft dan een eigen en nieuw project. Hij zal als 1-mansteam dit project op zich nemen. Het project zal geen raakvlakken hebben met huidige of voorgaande projecten. De student is zelf verantwoordelijk voor het gehele project. Alle aspecten binnen het project inclusief het onderzoek en het zorg dragen voor benodigde onderdelen zal tot het takenpakket behoren. Hierbij wordt er wel begeleidende hulp van de bedrijfsbegeleider aangeboden.

\subsection{Aard van de Opdracht}

Binnen een consultancy bedrijf is het lastig om voor afstudeerders een opdracht samen te stellen. Voor een afstudeerder is er te veel verantwoordelijkheid om diegene naar een klant te sturen met een bepaalde opdracht van die klant als afstudeer-opdracht. Ook is dat veelal niet de wens van de klant. 

Om afstudeerders toch een kans te geven bij Alten, biedt Alten projecten aan op zijn eigen kantoor. Wat Alten PTS vaker doet is het maken van 'proof of concept producten' die zij kunnen tonen aan potentiële klanten om de kansen bij specifieke problemen te kunnen weergeven. Zo zijn er al enige zogeheten demonstrators gemaakt door medewerkers en afstudeerders binnen Alten PTS.

Zo heeft Alten PTS verschillende algemene afstudeeropdrachten waarbinnen onderzoek kan worden gedaan. Voor dit project is de opdracht 'Veiligheid op de werkvloer' van toepassing.

De student heeft samen met medewerkers van Alten PTS gebrainstormd over een opdracht. Met de intentie om een opdracht te formuleren die interessant en leuk is voor de afstudeerder en die ook als potentieel product voor Alten zijn klanten interessant is. Deze opdracht moet wel binnen de scope van de opdracht 'Veiligheid op de werkvloer' vallen.

Deze brainstorm sessies zijn verlopen door de interesses van de afstudeerder te verkennen. In dit geval heeft de afstudeerder interesse in werken met sensoren en netwerken daarvan op een decentrale wijze. Hierna is er een bedrijfsbegeleider toegekend aan de afstudeerder om de opdracht specifieker te maken.

De opdracht is wel met zorg gekozen zodat Alten PTS er later baat bij kan hebben. Zoals beschreven is Alten PTS een bedrijf dat voor techniek staat. De opdracht moet dus ook aansluiten bij het profiel van Alten PTS.

\end{document}